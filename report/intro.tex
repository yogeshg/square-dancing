\section{Introduction}
\subsection{Premise}
In Slather, players vie for control of a 10cm x 10 cm board using pre-programmed cells that naturally grow by up to 1\% per turn, produce temporary trails of pheromones, and reproduce when their size reaches double their initial diameter of 1cm. Cells cannot collide with each other, and cells cannot collide with pheromones from another player’s cells. Each player can define a common algorithm for all their cells that has access to information about surrounding cells and pheromones, as well as a byte of persistent memory. The goal of Slather is to be the player that can reproduce the most and have the most cells on the board by the time no more reproduction can occur.\\


The game is made especially interesting because of the many different configurations that can be run. Adjustable variables include p, the total number of players, n, the number of starting cells for each player, d, the viewing distance for each cell, and t, the number of turns a pheromone lasts after being released by a cell. This means that players must be robust across multiple configurations that could range from anywhere between pairwise competitions with low view distance, short pheromone trails, and one starting cell—to nine-player competitions with high view distance, long pheromone trails, and many starting cells.

\subsection{Strategy Overview}
Our strategy divides the game into the early game, the mid game, and the late game. The game changes from one state to another based on the number of enemy and friendly cells detected around the player cell.
In the early game, our cell looks at all the cells around it and chooses the widest angle between two cells and moves in that direction. If there are no cells around, the cell simply stands still. As the cell reproduces, the daughter cells create distance between themselves based by moving into the largest angle. This leads to the creation of a cluster. Once the cells detect a certain number of cells around it, the cell infers that the map is starting to get packed and thus moves into the mid game.\\
\\
In the mid game, we classify cells as “border cells” and “interior cells”, and independently as “attackers” and “defenders”. The first category is inferred from the cells surrounding the player, as border cells typically have a large number of enemy cells nearby and interior cells have few, if any enemy cells nearby. The second classification of attacker and defender is made at the time of reproduction. We make a cell an attacker with 66\% probability and a defender with 33\% probability. When there are no enemy cells around, they both behave the same way, going toward the largest gap between friendly cells. The difference is when there are enemy cells nearby. Attackers tend to move toward enemy cells. Their purpose is two-fold. On the border, they push outward, thus expanding the cluster. In the interior, they detect enemy cells, and surround them, preventing them from growing inside the cluster, much like a white blood cell attacks a disease. The defenders behave differently. They tend to move away from friendly cells. Defenders on the border keep enemies out. If an enemy comes near, if the defender gets pushed in, it sees the friendly interior cells and starts pushing outward. In the interior, the defenders do not really have a purpose except to divide.\\
\\
The end game strategy is very similar to the mid game. The difference is that near the end, the gap between cells may not ever be large enough to move toward. So the cell randomly chooses different positions to move toward and if it succeeds in finding one, moves there.\\
\\
The result is the creation of a cluster that keeps growing. The goal of this strategy is to mark off space for our species and keep enemies out. 
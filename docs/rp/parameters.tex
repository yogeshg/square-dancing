\section{Parameter Tuning and Memory}
This section discusses the parameters we used and why we selected certain values:\\
\subsubsection{Soulmate Strategy Threshold}
This is the largest value of d below which we try to find soulmates. As we discussed in the medium d section, the score we get is around 5100 points with the medium d strategy. The score for the small d case monotonically decreases as we increase d (because it takes longer to find all soulmates). So, all we had to do to get the value for this threshold was to binary search over d for the point where the score using the small d strategy becomes 5100 points. The experimental value for this using our strategy turned out to be d = 908, which was the value we used for the threshold.\\
\subsubsection{Hexagonal Packing Thresholds}
As mentioned in the hexagonal packing section, our implementation did not allow for the first and last dancers to dance with hexagonal packing. This led to a decrease in score of $\frac{1}{row\ size}$ where row size is 40, because 40 dancers fit in a row. So we used square packing upto the maximum possible of 1600 (because 40 dancers per row and 40 per column) and then hexagonal packing upto 1840 (1600 + 6 more rows of 40).\\
\subsubsection{Target Score Percentage}
Since the target score is the theoretical best score that our algorithm can reach, it makes sense that our implementation may not be perfectly efficient and so the last batch may end up having less time to reach the target score.
We would rather aim for a slightly lower target score and reach it than have the last batch just entirely not reach the target score.  Decreasing the target score increases the score for the last batch equal to the number of batches per point decreased. This is because each batch tries to reach a score of one point less, which translates into all those turns getting saved up for the last batch.\\
We ran some experiments to try to find out what percentage of the target score we were able to reach. For the d values 2000, 3000 and 4000 we were able to exactly reach the target score. For d = 10000, we were able to reach 95 percent of the target score. We did not test for values above 10000, but we speculated that we would be able to reach around 95 percent of the target score. We ended up setting a conservative value of 90 percent, because it would be better to reach 90 percent of the target score than not reach it at all.\\
We also bel